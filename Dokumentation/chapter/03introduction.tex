%03introduction.tex
\renewcommand\thesection{\arabic{section}}
\chapter{Management summary}
\section{Problemstellung}
Je nach Untergrund, Bodeneigenschaften, Gel\"ande sowie Klima gedeihen in der Schweiz unterschiedliche Typen von W\"aldern. Seit einigen Jahrzehnten werden diese Typen von Experten erhoben und kartiert. Es wurden dabei verschiedene typisierte Waldstandorte festgelegt. Aktuell werden Karten, die im Auftrag der Kantone von Experten angefertigt wurden, nur in grossen Intervallen revidiert, wobei sie oft auch nicht fl\"achendeckend vorhanden sind (z.B. in den Kantonen GR, VS, BE).
Einer der Gr\"unde daf\"ur sind u.a. die hohen Kosten, die eine Analyse im Feld mit sich bringt.
Zudem ist die Erfassung und Nachf\"uhrung der Karten gepr\"agt von analogen Vorg\"angen, da die vorhandenen
technischen Ger\"ate und Programme f\"ur den Einsatz im Feld ungeeignet sind.
Daher muss von Hand Niedergeschriebenes im B\"uro oder von staatlichen Institutionen digitalisiert werden, bevor es an den
Arbeitgeber geschickt werden und sp\"ater auf kantonal isolierten Plattformen publiziert werden kann.

\section{Ziel der Arbeit}
Die Erfassung und Publikation von Waldstandorten sollte vereinfacht und beschleunigt werden. Dabei
sollen digitale Technologien eingesetzt werden wie Smartphone, GPS und Internet. Diese neuen
Instrumente sollen entsprechend geschulten Nutzern die Erfassung von Waldstandorten erm\"oglichen sowie \"offentliche und private Informationen in Form von Fl\"achen und Punkten. Auf einer Basis - Karte wird mittels GPS die eigene
Position angezeigt. Dar\"uber werden umliegende, bereits erfasste Waldstandorte, \"offentliche Fl\"achen anderer sowie die eigenen, privaten Fl\"achen dargestellt. Diese Fl\"achen k\"onnen Waldstandorte beschreiben oder aber zus\"atzliche Informationen \"uber den Standort beinhalten, z.B. eine speziell gekennzeichnete Beobachtungsfl\"ache.

\section{Ergebnisse}
Nach einer Evaluation eines Prototyps, erstellt mithilfe eines kommerziellen Produkts,
und der Erstellung von Mockups, wurde ein eigenes Webapp 'Waldmeister Outdoors' realisiert. Durch diese App kann die Arbeit der Experten erleichtert werden. Da die Waldstandort-Karte gleichzeitig im Web synchronisiert
ist, wird dar\"uber hinaus der Informationsaustausch unter allen Beteiligten erleichtert. Die Webapp wurde f\"ur mobile Ger\"ate optimiert und die gew\"unschten Funktionen wurden umgesetzt. Registrierte Benutzer k\"onnen Benutzerfl\"achen in Form von Polygonen direkt auf der angezeigten Map erstellen, mit zus\"atzlichen Informationen versehen und auf einem Server speichern.

\section{Ausblick}
Grosse Teile der Schweiz sind noch unkartiert, und viele Waldstandorte k\"onnten sich unter dem Einfluss der Klimaerw\"armung ver\"andern. Die kontinuierliche Beobachtung solcher Standorte ist Forschungsgegenstand und die Arbeit im Feld ist unerl\"asslich. "Waldmeister - Outdoors" kann im Berufsalltag sowie bei der Kommunikation mit Institutionen den Arbeitsfluss beschleunigen. Weitere Features wie die Verwendung von Plus Codes und offline-F\"ahigkeiten welche bei Verbindungsproblemen zum Einsatz kommen, bzw. Benutzerfl\"achen automatisch synchronisieren, sobald eine Verbindung besteht. Des weiteren bietet es sich an, dass sich User in Gruppen einklinken k\"onnen, um unter sich Benutzerfl\"achen zu teilen und zu besprechen, bevor sie ver\"offentlicht werden. Ebenfalls sollten erstellte Fl\"achen von registrierten Benutzern und deren Gruppen ver\"andert und gel\"oscht werden k\"onnen, nachdem sie erstellt wurden. $\newline$
"Waldmeister - Outdoors" hat das Potential in der Schweiz ein verbreitetes Tool zur Kartierung und Beobachtung von Waldfl\"achen zu werden und stellt eine bereits gefragte Erweiterung der beliebten "Waldmeister" App f\"ur mobile Ger\"ate dar.

\renewcommand\thesection{\thechapter.\arabic{section}}

\chapter{Teil 1 - Technischer Bericht}
\section{Einf\"uhrung}
\subsection{Problemstellung, Vision}
\subsection{Ziele und Unterziele}
\subsection{Rahmenbedingungen}
\subsection{Vorgehen, Aufbau der Arbeit}

\section{Stand der Technik}
\subsection{GIS-Browser}
\subsection{ArcGIS online}
\subsection{Defizite}

\section{Bewertung}
\subsection{Kriterien}
\subsection{Schlussfolgerungen}

\section{Umsetzungskonzept}
\subsection{L�sungsans�tze}
\subsubsection{Backend}
\subsubsection{Frontend}
\subsubsection{Datenbank}
\subsubsection{JavaScript Libraries}
\subsubsection{Python Pakete}
\subsubsection{Werkzeuge und Tools}
\subsubsection{Testing}

\section{Resultate}
\subsection{Zielerreichung}
\subsection{Ausblick und Weiterentwicklung}
\subsection{Pers�nliche Berichte}
\subsection{Danksagungen}
\clearpage
\pagebreak


\chapter{Teil 2 - SW-Projektdokumentation}
\section{Vision}

\section{Anforderungsspezifikationen}
\subsection{Must-Haves}
\subsection{Optional}
\subsection{Use-Cases}
\subsection{Funktionale Anforderungen}
\subsection{Nicht-funktionale Anforderungen}
\subsection{Weitere Funktionen und Anforderungen}
\subsection{Detailspezifikationen}

\section{Analyse}
\subsection{Klassendiagramm}
\subsection{Domain Modell}
\subsection{Objektkatalog}

\section{Technologien}
\subsection{Django}
\subsection{VueJS}

\section{Design}
\subsection{Architektur}
\subsection{Objektkatalog}
\subsection{Package Struktur}
\subsection{Sequenz-Diagramm}
\subsection{UI Design}

\section{Implementation und Test}
\subsection{Implementation}
\subsubsection{Technische Implementation}
\subsubsection{API und Datenquellen}
\subsubsection{Login}
\subsubsection{Kartenmaterial}
\subsubsection{etc.}

\section{Automatische Testverfahren}
\subsection{Jasmine}
\subsection{Karma}
\subsection{Travis}
\section{Manuelle Tests}
\subsection{Testf�lle}
\subsection{Ziel}
\subsection{Reproduktion}

\section{Resultate und Weiterentwicklung}
\subsection{Resultate}
\subsection{M�glichkeiten der Weiterentwicklung}
\subsection{Vorgehen}

\section{Projektmanagement}
\subsection{Allgemeines}
\subsubsection{GitHub}

\subsection{Meilensteinplanung}
\subsection{Releases}
\subsection{Issues}
\subsection{Prototypen}
\subsection{Prozessmodell}
\subsubsection{Scrum}

\subsection{Aufwandsch�tzung}
\subsubsection{Zeitplan}
\subsubsection{Projektplan}

\subsection{Risiken}
\subsubsection{Deadlines}
\subsubsection{Technologien}
\subsubsection{Frameworks}
\subsubsection{Mobile Limits}

\section{Projektmonitoring}
\subsection{Soll-Ist-Zeitvergleich}
\subsection{Code Statistics}

\section{Softwaredokumentation}
\subsection{Installation}
\subsection{Tutorial, Handbuch}
\subsection{Referenzhandbuch}







